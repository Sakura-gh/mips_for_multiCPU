本次实验由于疫情影响,只能利用模拟器进行软件方面的验证,而硬件方面的验证只能等到8月份返校之后进行。在软件部分,主要涉及到mips汇编程序设计的编写,由于受到可用指令数量的限制,以及汇编代码的不易读性,因此具体的实现过程中也遇到过不少困难。其中遇到过最多的问题,就是变量和堆栈所造成的。当不同函数之间相互跳转的次数多了以后,很多情况下这些函数又会用到相同的寄存器作为临时变量,因此特别需要注意将变量的值保存到堆栈里,并当函数返回时恢复变量的值。除此之外,如何设计程序的整体框架;如何存放坦克、障碍物、子弹的数据并实时更新与绘制;如何判断子弹击中的是哪个障碍物;如何判断坦克是否与障碍物相撞...一系列的问题都需要花费大量的时间去分析和处理。\\

这次汇编大程锻炼了我的细心和耐心,有了第一次汇编project作为基础,这次的大程设计起来也没有非常的困难,写了一个多星期就把软件部分给写完了,在这一个多星期里,绝大多数时间也还是花在debug上。除此之外,尽管程序在模拟器上成功运行,但是否能够在硬件上调试成功目前尚未知晓。还是期待能早日返校做物理验证吧。\\