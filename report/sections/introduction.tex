\subsection{设计背景}
本项目为计算机组成综合性课程设计——简易SOC或微控制器应用设计(软硬协同设计)的成果。项目的出发点是基于计算机组成Exp03~Exp12已经搭建完毕的单周期或多周期CPU,在支持的18+条扩展指令的基础上,搭建一个小型应用程序,需要有基本的IO交互与运行逻辑,具备一定的设计意义。\\

这里采用支持18+指令的多周期MCPU,并扩展了VGA显示和PS2键盘交互的功能,设计了一款简单却有意义的坦克大战小游戏——Only One Shot。该游戏设计的基本思想,是利用简单的键盘交互来实现用户对坦克上下左右移动的控制,以及子弹的发射,并通过刷新显示在屏幕上,这一部分检验了vga显示和ps2的交互功能。游戏将每隔一定时间在随机位置生成障碍物,这一部分将由对计数器counter的读取来实现。每轮游戏都将记录当前游戏得分以及多轮游戏的最高分,这一部分将由七段数码管作为显示设备。除此之外,如何验证子弹是否击中障碍物、坦克是否遇到障碍物、障碍物是否出界...这些都考验了复杂的逻辑设计能力。\\

值得一提的是,本项目既然叫做"Only One Shot",顾名思义,它的特点就是:坦克只有一颗可利用的子弹。只有当该子弹击中障碍物或用户主动放弃当前已经击出的子弹时,才能重新使该子弹出膛来射击另一个方向的障碍物。为了增加难度,子弹并不是立马发射出去的,我单独设计了一个子弹从预备发射到出膛的动画过程,也就是说,用户在发射子弹之前,务必要估计自己在被障碍物撞上之前是否能够成功发射子弹。\\
% 与大多数游戏类似,Only One Shot采用'w', 's', 'a', 'd'这四个按键来实现坦克的上下左右移动,并利用按键'o'来实现子弹的发射。

通过该项目设计,能够较好地检验多周期CPU的功能,并扩展了外部I/O设备接口,能够锻炼复杂的汇编程序设计能力,进而加深对计算机运行过程的认知。独有的only one shot设计理念,也算是在简陋的框架环境下对创新的一点尝试。希望借助本项目,能够帮助自己进一步地理解计算机组成原理,提高汇编程序设计水平。\\

\subsection{国内外现状分析}

为了更好地认知计算机组成及其实验课在计算机专业上的重要性,这里查阅了一些文献和资料,并对MIT、UC-Berkeley、Stanford大学、CMU这四所大学相关课程的实验情况进行了简要地总结。这些信息都源自这些大学相关课程的最新课程网站。\\

通过对多个美国一流大学在相关课程方面教学内容的调查,基本可以总结如下:除MIT的课程内容较偏底层硬件外,其他三所学校相关课程的教学内容和实验内容基本类似,其教学理念和教学思路也非常相似,基本上都是按照“C语言程序→汇编语言程序→机器目标代码→处理器结构”为主线组织内容,都是站在计算机系统的高度来阐述计算机硬件系统的结构和设计思想。\\

\begin{figure}[h]
  \begin{overpic}[width=\columnwidth]{aborad.png}
  \end{overpic}
  \caption{美国部分大学计算机专业相关课程实验教学基本情况
  }\label{fig:colorFre}
\end{figure}

由于浙大的计算机组成教材采用国外出版的“Computer Organization and Design”一书,因此在课程设计和思想方面,均与国外一流大学保持同步。此外,国外高校也积极采用在自行设计的CPU上运行汇编程序project的方式来检验学生学习的能力。大多数项目设计基本思想都比较接近,经过调查,虽然同样有坦克大战类型的游戏出现,但赋以“Only One Shot”特性的汇编程序是本项目所特有的。\\

\subsection{主要内容和难点}

这里将简要介绍本设计要完成的主要内容、功能、技术要求和目的,以及实现的重点难点。

\subsubsection{主要内容}
1、SOC基本框架\\
2、支持18+指令的多周期MCPU\\
3、实现vga和ps2的I/O接口\\
4、实现Only One Shot完整功能汇编程序,`.asm`文件\\

\subsubsection{设计功能}

mips汇编文件能够在多周期MCPU上成功运行,实现只带有一发子弹的坦克大战游戏。游戏规则如下:

\begin{itemize}
    \item [1)]
    每隔一定时间将会在随机位置产生障碍物,坦克可以选择发射子弹或不发射,并且可以上下左右自由移动。
    \item[2)]
    用户只有一发子弹,在之前射出的子弹尚未击中障碍物,而用户又重新按下发射键的时候,之前发射的子弹将会消失,并重新被装载到坦克车身上,开始出膛射击。
    \item[3)]
    如果子弹击中障碍物,则该子弹会与障碍物同时消失,用户将会获得1分。
    \item[4)]
    如果障碍物到达并超出边界,则用户将会被惩罚而扣2分。
    \item[5)]
    如果障碍物撞上坦克,则游戏结束。
    \item[6)]
    计分板将会实时更新当前用户得分,以及历史最高得分。
\end{itemize}

注意,子弹出膛需要一定的时间(这里专门设计了子弹出膛的动画)。因此在障碍物与坦克很接近的情况下,即使用户按下子弹发射键,如果在子弹成功出膛前,坦克就已经与障碍物相撞了,则游戏同样结束。故用户需要额外判定当前时机是否适合发射子弹,这也是本游戏的特色之一。\\

\subsubsection{技术要求与目的}

本设计要求掌握基本的verilog语法,具备一定的mips汇编能力,并能够熟练地构建ise工程,进行仿真、调试与物理验证。

\subsubsection{重点与难点}

本设计虽然思路比较清晰,但在具体实践的过程中,也有不少难点需要克服。

\begin{itemize}
    \item [1)] 尽管多周期CPU支持18+条指令,但实际上未能满足项目开发的实际需求,需要额外扩展指令,或者以软件的形式加以实现。
    \item[2)] 由于障碍物是随机位置生成的,如何获取随机数,并恰当地转化为符合屏幕大小的坐标是一个难点。
    \item[3)] 每次生成一个新的障碍物,都需要有指定的地址来保存它的位置信息,否则下一次刷新画面时,该障碍物就会丢失。  
    \item[4)] 每次障碍物由于被击中或超出界面而消失,都需要在指定的地址上抹除该障碍物的位置信息,否则下一次刷新时,该障碍物依旧会出现。
    \item[5)] 由于障碍物的数量每时每刻都在变化,位置也在时刻更新,因此如何去遍历更新这些数据,在什么地方删去旧的记录,什么地方插入新的记录,也是一个难点。
    \item[6)] 子弹每次的发射位置都必须是坦克的"弹仓"位置,且需要有一个从坦克尾到坦克头发射的过渡动画。
    \item[7)] 坦克需要能够随着ps2键盘的按下进行上下左右移动和子弹的发射
    \item[8)] 如何判断子弹是否与障碍物相遇,坦克是否与障碍物相遇,障碍物是否出界,需要利用多少个点的相对位置来判断,也是一个难点。
    \item[9)] 如何在刷新的时候实时处理坦克、子弹、障碍物在屏幕上的画面。
    \item[10)] 如何利用七段数码管来统计当前得分,并同时保存并显示历史最高分。
\end{itemize}



% 1、Only One Shot游戏由坦克(tank)、障碍物(obstacle)和